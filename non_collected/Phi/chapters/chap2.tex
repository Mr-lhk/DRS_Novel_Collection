\chapter{\texorpdfstring{\nphi}{phi}}

\epigraph{Odi et amo. Quare id faciam, fortasse requiris.\\
我恨,我又爱。你问为何如此?我也不知。\\
Nescio, sed fieri sentio et excrucior.\\
但我感到如此——而我因此痛苦。
}
{Catullus\\
卡图卢斯}

\nphi 很怨恨。起码,现在感到很怨恨。

\vspace{8ex}

远在一年之前,那是\nphi 刚进入高中时,是\nphi 人生中的最低谷——惨淡的中考成绩,如一纸判决,宣言着她的失败,褪色了她的人生。

秋的天总是灰白的,像是被脏水冲刷过的纸。迈入新校园的大门,没有喧闹的笑声,也没有想象中的青春气息。风掠过树梢,麻雀展翅飞远。教室的灯光刺眼,如寒霜般雪白,冷冷地注视着\nphi ,让她愧怍得无地自容。

\vspace{8ex}

那是让人丧失一切希望的日子。

\vspace{8ex}

但她遇见了\nvphi 。

独坐在沉寂空旷的教室里,寒风呼啸着从大开的窗户里咆哮而过,卷起试卷漫天飞舞。纸页在空中旋转,又如冬雪飘落。

刹那间,风停云淡。

在被撕开的云层间,金黄的光芒从天顶直射而下——刺眼,但温暖。

那一束神圣的光芒不偏不倚,恰落在门口,落在那一个身影身上。

\nvphi 正站在那里。

她逆光而立,黑色短发被金色的阳光笼罩,信念的碎光在眼神中闪烁,娇小的身躯在阳光中发亮——那不属于这个世界的坚毅与温柔,让整个世界为她开路。

她微微侧头,脸上露出淡淡的笑容,却足以将\nphi 的世界重新点亮。

那一刻,一种从未有过的情感在\nphi 心中漫生——仿佛心中漫长的阴影在她的光辉下退却。

风又起,却不再刺骨。光仍在,覆在\nvphi 的身上。

\nphi 能听到,自己的心,在那片明亮中轻轻地颤抖。

\vspace{8ex}

\nvphi 成为了\nphi 生活中不可或缺的一部分。

她们开始坐在一起——吃饭、学习、聊天,几乎形影不离。

课间,是两人休息的时间。清风拂过桌面,\nvphi 会伸手压住那几页被风吹起的纸张。
那手指修长且纤细,骨节分明,虽微凉,却让人感到安心。\nphi 总是看着那只手愣神,然后忽然意识到什么,忙低下头,假装在写些什么。

午后,是两人共处的时光。并肩走过操场,影子在地面交织,风掠过发梢,带着摩挲的声响。\nvphi 总是轻声说笑,\nphi 只是点头应和,但在那声音的起伏间,却总能察觉到\nphi 心底某种温柔的悸动。

放学,是两人难别的时刻。天光渐渐暗下,夕阳也流逝而去,在\nvphi 的肩头拖下一条浅淡的光晕。那光一点点移动,顺着她的轮廓流淌。\nphi 望着那痕迹,不舍得打扰这温柔的瞬间。

时间就这样缓缓流过。那是\nphi 有\nvphi 相伴于身边的日子——早晨一进教室,就能看见心中的那抹光;有人会关心她的身体,送上手边的一杯水;听着\nvphi 的畅快的笑声在风中回荡,带着心中的温暖。

这些天,\nphi 想了很多。在夜里,偶尔梦回那间被光照亮的教室,梦见\nvphi 转头微笑的样子——梦里,风很轻,光很暖,而\nphi 的心,也在光里慢慢融化。

\vspace{8ex}

而这一切,从那一天起,戛然而止。

\vspace{8ex}

那次与\nPhi 的偶遇后,光芒,似乎不再照向\nphi 。

起初,只是些微不足道的变化。只是午间食堂路上的几分寂寞,只是放学后悄然溜走的身影。

\nphi 不断地告诉自己——那只是\nvphi 的事情,她无权干涉。

可身旁的那份空白,却随着时间慢慢扩散,光线在尘埃中消散,淡而无声。

课堂上,\nvphi 的笑声依然回荡,只是那笑声里多了一份陌生的轻快。她的目光总是不自觉地追随着笑声而去,却在\nvphi 身旁看见\nPhi 的那一刻,僵在半空。两人的谈笑像被阳光笼罩的风景——明亮、温柔,但光芒中,却没有了\nphi 的身影。

阳光依旧从天顶洒落,只是那道光的中心,不再有她。风从窗外涌入,桌上纸页翻飞。\nphi 伸手想去抚平,却在触到那张空白的桌面时,忽然怔住。那里原本放着两人互相交换的纸条,贮存着\nvphi 的字迹与笑声——如今,那里空空如也,只剩一层薄尘,轻轻覆在那曾经美好的回忆之上。

\vspace{8ex}

夜里,\nphi 辗转难眠。风从忘记关严的窗缝里钻进来,拂动窗帘,寒气逼人。她闭上眼,那束光仍在眼前——那是她们的初见,\nvphi 逆光而立,笑容一如往常。\nphi 伸出手,试图触碰再次去触摸那温暖的光辉,那光却毫不留情地碎了,如同水面上的薄冰,一点点破碎、坠落,最终一无所有。

从梦中惊醒,天已微亮。冷光从地平线上斜射而下。她拿起手机,\nvphi 的窗口已沉寂了许久,却忽然弹出一条更新——那是 \nvphi 与\nPhi 的欢声笑语。手指在视频上悬停了许久,又止步。心里一阵疼痛,无法掩饰,像世界的坍塌,无从躲避,唯余无言。在内心最阴暗的角落,一缕缕黑雾已开始蔓延。

\nphi 不知道自己究竟在怨恨什么。到底是怨恨那道光离开了她,还是怨恨自己,从一开始就只是那光的影子,不敢主动追随。

\vspace{8ex}

孤独的夜愈加漫长。

每当风掠过窗缝,\nphi 的心中都会响起那熟悉的笑声——轻柔,却又无比锋利,在心头割裂出一道又一道的痕。她开始厌恶那笑声,因为它不再属于她。

更准确地说,那笑声里,再也没有她的位置。

教室中,她仍旧坐在原来的地方,指尖一遍遍描摹着那一道道字痕,那是\nvphi 的笔尖无意在桌上刻下的印,却仿佛刻在了\nphi 的心间。那一点点的划痕,在白炽灯照耀下微微反光,温度早已散尽。

往事涌上心头,在心中激起一阵涟漪——如果能让一切都像从前那样,驻足在那些美好的时光中,该有多好。

越来越多的回忆侵蚀着\nphi 的思绪。\nvphi 的笑、她的语调、她无意间的侧颜\dots 每一幕都飘落在了回忆与现实的分界线上,飘渺而不可及,却显得那么真实。她甚至开始分不清,这些记忆,究竟是保存下来的,还是她用渴望伪造出来的。

夜色越深,记忆越明亮。光仍在,只是照得她眼痛,无法直视,隐于心间。

她开始感到害怕。害怕那道光彻底消失、弥散,害怕连记忆都疏远到背叛她。她开始跟随——在教室的尽头、在走廊的拐角、在食堂的桌旁——她假装路过。看着\nvphi 与\nPhi 并肩而行,听着她的笑声渐远,心底被一点点撕裂。终于,下定了那个曾不敢触及的决心。

那早已不仅仅是嫉妒,而是内心深处的渴求。她渴求的不是\nvphi 的笑,也不是她的陪伴——她渴求的,是那份被光照亮的确定感。

她开始做梦。梦里,光回来了。\nvphi 又站在那间教室门口,逆光而立。她试探着伸出手,这一次,光,完好如初。她再一次触碰到了那温暖,也触到了那份不受袭扰的安宁。

她在梦中轻声说:

\begin{quote}
“别走。”
\end{quote}

光没有回答。但她知道——这一次,光独属于她。
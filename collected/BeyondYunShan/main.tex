% 本作品保留所有权利
\documentclass[../../Anthology.tex]{subfiles}

\title{云山尽处}
\author{佚名}
\date{2025/11/20}

\begin{document}

\ifSubfilesClassLoaded{\maketitle}{\chapter{云山尽处}}

佥事犹豫了很久,还是决定将那只小鸟埋在宅子中的老槐树下。

盖上最后一捧泥土,他又一个人在树下站了很久。不知何时下起了雨,冷雨淅沥,青色的官服早已湿透,紧贴其身,勾勒出清瘦而挺直的脊梁。水珠从他紧抿的唇角划过,他恍然未觉。

就在刚刚,他亲手埋葬了他的理想主义。

\bigskip

生于当地大户,从儿时起,他的身边就充斥着这样的话:“考取功名”、“走多数人都在走的路”、“\dots ”每当听到这样的话时,他都会默默攥紧手中的地图册藏到身后。或许是不够坚定吧,抑或是想满足他人的期待,当母亲问他是否想去游历山川时,他说:“儿想做官。”

这之后的日子,加冠,科举,做个说大不大说小不小的佥事。他记不起多久没再翻开过少年时的笔记了,可依然会在梦中勾勒书中独到的每一座高山、每一条河流。他时常和自己说:“这是一条正确的道路不是吗?”直到那天,一只云雀落在他的窗檐上。

它灰褐的羽翼像是山石的颜色,清冷的啼叫像是清澈的小溪,明亮漆黑的眼睛像初雪洗涤过的黑曜石。它瘸了一只腿,用单脚跳着。他把它安置在笼子里,每日给它擦药、喂水。一段日子后,它似乎好起来了,但似乎总是闷闷不乐。直到一天清晨,它静静地躺在笼子里,不复生机。

佥事这才意识到,那云雀不是他,鸟儿要飞翔的。那是曾经的他。

可是已经于事无补,他的一生都将困于当下的这一方天地。

\bigskip

院门口出现了一个游人,他看起来像是二三十岁,身上背着行囊,眼睛中闪着兴奋的光芒。他们的眼神交织,他忽然觉得这游人有些眼熟,但又想不起来何时见过,内心传来隐隐的苦涩。

“请问我有什么可以帮您的吗?”游人的声音带着少年气。

“我在陪我刚离开的鸟儿。”佥事颓唐地答。他自己也说不清楚,他为什么会如此直截了当地把心事吐露给一位素昧平生的游人。

游人听了佥事的话,愣了半晌,随即正色道:

“它度过了快乐的一生。”

“这就足够?即使死去?”

“即使死去。”游人望向远方,“死亡只是从天空回归大地,成为高山上的一捧土,化作汇入江海的细流。飞渡湖岳,遍及九洲。”

佥事只是怔愣地听着,他心中的那种感觉愈发强烈了,他回忆起了这个游人的脸庞。他出现在他的脚步遍及山川的梦里。听着听着,泪水不觉间浸湿了他的双眼。他终于知道为什么这个年轻的游人的面孔与他的少年时期如此相似了。

“您还好吗?”游人满眼焦急无措。

“没事。”佥事突然笑了。“你能多给我讲讲你的故事吗?”

那天,佥事忘记了公务,只是听那位年轻的游子讲了很久很久。天台、泰山、湘江、锦江\dots 他看着游人雀跃的神情,在他生动的讲述中,似乎回到了从前,似乎他真的踏上了曾经书中谈到过的土地,遍览了一寸寸风光。他看到了他人生的另一种可能。

他仿佛也作了一回自由自在的云雀。

雨停了,游人在一个夕阳西下的时分离开了,佥事本想留他几日,他只笑着摆摆手说他要去前方的一座山。佥事目送着他的背影,直到他的身影消失在一个岔路口。“谢谢你,替我走过了。”佥事默默地想着。

他走向了一条人迹罕至的路——走向云山尽处。

“我也要去走我的路了。”佥事转过身。

数年后,佥事已经忘记了那一天具体的情景,只像梦境一般模糊地记得那天目送远去的,背着行囊的背影。已然白发苍苍的他,机缘巧合地翻到了一卷书。仅仅看了一眼,他的双手便止不住地颤抖。

\smallskip

《徐霞客游记》

\smallskip

“如果有来世,我也应当做徐霞客。”

年逾古稀的徐弘祖这般呢喃道。

\bigskip

\begin{center}
    \small
    \color{gray!70}
    本作品保留所有权利。
\end{center}

\end{document}
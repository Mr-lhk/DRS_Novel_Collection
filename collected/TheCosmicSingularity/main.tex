\documentclass[../../Anthology.tex]{subfiles}

\title{宇宙奇点}
\author{宋哲凡}
\date{2025/11/18}

\begin{document}

\ifSubfilesClassLoaded{\maketitle}{\chapter{宇宙奇点}}

我醒过来,已经是晚上十点钟——虽然在这里不存在时间的概念,但退休前的最后一晚,我仍希望一切如常。简单洗漱,我整了整衣襟,转身走出宿舍的门。

去往办公区的路上,时不时会有几个小年轻,拎着大包小卷,抱怨着:“怎么又要换站啊,这是今年第几个了\dots ”见了我,马上又毕恭毕敬致意:“沐站长好!”我微微点头,望着那疲惫却带着光的眼睛,心中不由得涌起一丝感慨。回想起自己刚到这座奇点空间站时——当然了,以前的记忆出于工作需要,连我自己都想不起来,这是我脑中现存的第一段记忆——“中遂科”三个红黑配色的大字在穹顶下闪耀,我还记得那种感觉,那种梦想成真的激动。入职培训,我现在还记得那个考官的脸。“\dots 我叫洪英,是你们的持剑者培训人\dots ””“dots ‘中遂科’,全名‘中止犯罪既遂科’,以中止犯罪既遂为第一,也是唯一目标\dots ”“\dots 这里,是宇宙奇点,时间在这里毫无意义,你们在过去、现在,也在将来\dots ”“\dots 职位分为持剑者与灵视者,持剑者身先士卒,灵视者在站内远程辅助\dots ”“\dots 作为持剑者,你们的任务很简单,系统会将你们的意识传送到案件的受害人身上,通常在行凶发生的几分钟之前,你们只需尽你们所能,帮助被害人逃脱被杀的结局\dots ”

回过神来,我嘲弄地轻笑:“说着容易做着难。”四十年,经手的案子大大小小几百件,虽然拥有无限重来的机会,但每次失败的痛,那可都是刻骨铭心啊,自己这把老骨头,也该跟着这十三号站,一起退休了。站内人已走空,我正准备关闭系统电源。“嗡!嗡!”象征报案的警示灯亮起。我心里直犯嘀咕,不是待废站点不给新案么\dots ?看了眼任务简报,一场普普通通的凶杀案,没有灵视者,只是做不到中途断开,只要任务完成,无人死亡,也能自动断开。我没多想,戴上设备,向后微躺,闭上双眼。

“啊\dots 靠!”随着全身的一阵剧痛,我从冰冷的水泥地上醒来。这是一间全封闭的屋子,冰冷,潮湿,似乎是个地下室。屋内没有灯,我站起身,发现左脚被镣铐锁住,勉强触及四壁。我敲击四壁,其中三面实心,剩下一面似乎较薄,墙后空心。我四处摸索,在墙角找到一把刀——奇怪,如果是有人将受害者囚禁,会留下如此明显的逃生方法么?正当我要凿向那面墙时,一盏昏暗的灯光亮起,那堵墙向两边移开,对面站着一个同样惊慌失措的男人,他的房间与我完全对称,眼神因极度恐惧几近疯狂,十指鲜血淋漓。

一个经变声器处理的男声响起:“杀死对方,才能活着出来。”

我当然不想这么干,但他已经扑了上来,慌乱中我用刀刺中了他的脖子。

眼前一黑,我再次醒来,水泥地,墙,灯,我的双手鲜血淋漓\dots

\bigskip

“尸检报告。”

“沐思,男,六十岁,中遂科第十三奇点空间站站长。记录显示其于二月十三日晚十一点半接入系统,尸体于二月十四日午十二点被前往工程队发现,身体因极度痛苦而扭曲。根据穿越机制,标准世界时十二个半小时,约为案件时间一百二十五年。死因初步断定为长时间经理案件却无法退出造成的脑死亡,系自身操作失误导致,性质为意外死亡。”

一艘无牌飞舰内,老者举起手中的报告。

“行,就这样改,去吧。”

身旁的侍从退出房间。

老者的嘴角,勾起一个弧度。

那是大仇得报的释然。

“老朋友,好久不见。”

\end{document}
